%----------------------------------------------------------------------------------------
%	PACKAGES AND THEMES
%----------------------------------------------------------------------------------------
\documentclass[aspectratio=169,xcolor=dvipsnames]{beamer}
\usetheme{Simple}


\usepackage{graphicx} % Allows including images
\usepackage{booktabs} % Allows the use of \toprule, \midrule and \bottomrule in tables
\usepackage{ulem} % delete lines
\usepackage{blindtext}
\usepackage{hyperref}

\hypersetup{
    colorlinks=true,
    linkcolor=blue,
    urlcolor=cyan,
    filecolor=magenta
}

%----------------------------------------------------------------------------------------
%	TITLE PAGE
%----------------------------------------------------------------------------------------

% The title
\title[short title]{DOMjudge Jury 指南}

\author{張源峷}
\date{}

\begin{document}


    \begin{frame}
        \titlepage
    \end{frame}

    \begin{frame}{目錄}
        \tableofcontents
    \end{frame}

%------------------------------------------------
    \section{說明}
%------------------------------------------------

    \begin{frame}{說明}
        \centerline{本Manual簡單介紹如何使用DOMjudge的裁判介面評測繳交程式,以及簡單功能說明}
    \end{frame}

%------------------------------------------------
    \section{創建帳號}
%------------------------------------------------

    \begin{frame}{創建帳號: 創建隊伍(team)}
        \begin{itemize}
            \item 進入 \href{http://127.0.0.1}{judge} 並點擊右上角的Login按鈕
            \includegraphics[height=30pt]{programming/domjudge_figure/1_login.png}
            \item 登入公用帳號 \newline
                帳號: public \newline
                密碼: public2022
            \item Teams/Add new team
            \item 新增隊伍 \newline
                Team name: 隊伍名稱\newline
                Display name: 顯示在Scoreboard上的名稱\newline
                Category: 選Observers\newline
                勾選Add user for this team \newline
                Username: 登入用帳號 \newline
                按Save保存
        \end{itemize}
    \end{frame}
    
    \begin{frame}{創建帳號: 設定使用者}
        \begin{itemize}
            \item 按左上角Logo回到Jury interface
            \item 進入Users
            \item 點擊剛才輸入的Username右邊的\includegraphics[height=15pt]{programming/domjudge_figure/icon_edit.png}按鈕(或是點進去後按Edit)
            \item 編輯使用者
                \begin{itemize}
                    \item 設定Password\textbf{*一定要設! 空密碼無法登入}
                    \item 設定角色: 如下圖所示\\
                    \includegraphics[height=90pt]{programming/domjudge_figure/role.png}
                    \item 按Save保存
                \end{itemize}
        \end{itemize}
    \end{frame}
    
    \begin{frame}{創建帳號: 登入及確認設定}
        \begin{itemize}
            \item 點擊右上角的頭像 $\rightarrow$ Logout \\
            \includegraphics[height=60pt]{programming/domjudge_figure/logout.png}
            \item 用剛才的帳號密碼登入
            \item Public帳號暫時僅作創建帳號使用,之後操作請儘量用剛才創建的帳號
            \item 間諜同學\underline{陳筱玲} (同學帳號測試用) \\
                帳號: Student1 \\
                密碼: stu1
        \end{itemize}
    \end{frame}
    
%------------------------------------------------
    \section{介面說明}
%------------------------------------------------

    \begin{frame}{介面說明: Navigation Bar}
        \includegraphics[width=400pt]{programming/domjudge_figure/nav_l.png} \\
        常用功能:
        \begin{enumerate}
            \item DOMjudge Logo: 進入DOMjudge Jury interface
            \item submissions: 管理繳交狀況
            \item scoreboard: 進入計分板
            \item team: 隊伍模式
        \end{enumerate} \par
        
        \includegraphics[height=25pt]{programming/domjudge_figure/nav_r.png}\\
        常用功能:
        \begin{enumerate}
            \item 頭像: 帳號動作(開關通知、登出)
            \item 獎盃: 當前比賽
            \item 時鐘: 當前比賽剩餘時間 \sout{防疫大將軍}
        \end{enumerate}
        
    \end{frame}

    \subsection{計分板}
    \begin{frame}{介面說明:計分板(Scoreboard)}
        \centerline{\includegraphics[height=100pt]{programming/domjudge_figure/scoreboard.png}}
        \begin{enumerate}
            \item 隊伍名稱
            \item 分數(題數|總Penalty) \\
                各題Penalty計算方式: AC時間(分鐘) + 罰時$\times$錯誤次數
            \item 題目簡稱
            \item 正確人數 / 錯誤人數
        \end{enumerate}
        
        *左上角有Filter可以使用
    \end{frame}
    
    \subsection{上傳紀錄}
    \begin{frame}{介面說明:上傳紀錄(Submissions)}
        \centerline{\includegraphics[width=400pt]{programming/domjudge_figure/sub_title.png}}
        \begin{itemize}
            \item result: 評判結果 \\
            \begin{itemize}
                \item \textbf{\textcolor{gray}{JUDGING}}: 評判中
                \item \textbf{\textcolor{OliveGreen}{CORRECT}}(AC): 答案正確
                \item \textbf{\textcolor{red}{WRONG-ANSWER}}(WA): 答案有誤
                \item \textbf{\textcolor{red}{TIMELIMIT}}(TLE): 解題程式執行時間過久
                \item \textbf{\textcolor{red}{RUN-ERROR}}(RE): 解題程式遇到執行時期錯誤
                \item \textbf{\textcolor{red}{COMPILER-ERROR}}(CE): 解題程式編譯失敗
            \end{itemize}
            \item test results: 各測資解答狀況
        \end{itemize}
        
        *左上角有Filter可以使用 \\
        *可以點進各別紀錄查看詳細(接下頁)
    \end{frame}
    
    \begin{frame}{介面說明:上傳紀錄(Submissions)}
        詳細統計 \\
        \centerline{\includegraphics[height=150pt]{programming/domjudge_figure/sub_info.png}}
        \begin{itemize}
            \item Rejudge: 重新評測此題(通常測資有改時用)
            \item View source code: 可以看到程式碼檔案並修改/測試
        \end{itemize}
    \end{frame}
    
    \subsection{隊伍模式}
    \begin{frame}{介面說明:隊伍模式}
        *此畫面與學生看到的畫面一致(學生無Jury按鈕) \\
        \includegraphics[height=25pt]{programming/domjudge_figure/team_navbar.png}
        \begin{itemize}
            \item Problemset: 該比賽的題目總覽
            \item Jury: 回到Jury Interface
            \item Submit: 上傳解答程式
        \end{itemize}
        
        *可以在此上傳程式測試題目
    \end{frame}
    
%------------------------------------------------
    \section{出題指南}
%------------------------------------------------

    \subsection{如何出題}
    \begin{frame}{如何出題}
        \begin{enumerate}
            \item 寫好題本: 將敍述說明清楚、確定輸入/輸出格式
            \item AC Code: 用自己預設的做法寫出正確答案
            \item 生產測資及答案檔\\
            \begin{itemize}
                \item 用random或手動方式產生測試資料 \\
                    建議順序: 範例測資 $\rightarrow$ 小測資 $\rightarrow$ 大測資(邊界測資)
                \item 將測資丟進AC Code,產生答案
            \end{itemize}
            \item 上傳至DOMjudge並測試題目
        \end{enumerate}
    \end{frame}
    
    \subsection{上傳題目}
    \begin{frame}{上傳題目}
        zip壓縮檔格式: \\
        \includegraphics[height=120pt]{programming/domjudge_figure/probzip_tree.png} \\
        \begin{itemize}
            \item problem.pdf: 題本
            \item data/sectet/*.in: 測資
            \item data/secret/*.ans: 答案
        \end{itemize}
    \end{frame}
    
    \begin{frame}{上傳題目}
        \begin{itemize}
            \item 至 Jury interface/Problems \\
            \includegraphics[height=100pt]{programming/domjudge_figure/prob.png}
            \item 選擇比賽 $\rightarrow$ 瀏覽壓縮檔(壓縮檔檔名會作為該題名稱)
            \item 按Upload上傳
            \item 出現如\includegraphics[height=1.5em]{programming/domjudge_figure/prob_ok.png}等訊息即為成功
        \end{itemize}
    \end{frame}
    
    \begin{frame}{上傳題目}
        (上傳完應該會停留在該題畫面,或可以至 Jury interface/Problems 再點選該題)\\
        \centerline{\includegraphics[height=150pt]{programming/domjudge_figure/prob_edit.png}}
        \begin{itemize}
            \item 按Edit編輯
            \item 調整時間限制等參數
            \item 按Save儲存
        \end{itemize}
    \end{frame}
    
    \subsection{題目範例}
    \begin{frame}{題目範例}
        \begin{itemize}
            \item 題本範例(模板): hw\_problems/problem\_template.tex
            \item AC Code 範例(ac.cpp) \\
                \vspace{0.5em}
                \includegraphics[height=150pt]{programming/domjudge_figure/ac.png}
        \end{itemize}
    \end{frame}
    
    \begin{frame}{題目範例}
        \begin{itemize}
            \item 亂數測資產生器範例(rand.cpp) \\
                \vspace{0.5em}
                \includegraphics[height=150pt]{programming/domjudge_figure/rand.png}
            \item 另參考C++亂數如uniform\_int\_distribution
        \end{itemize}
    \end{frame}
    
    \begin{frame}{題目範例}
        \begin{itemize}
            \item 亂數產生測資及產生解答 \\
            \vspace{0.5em}
            \includegraphics[width=400pt]{programming/domjudge_figure/run_data.png}
        \end{itemize}
    \end{frame}
    
    
    
\end{document}